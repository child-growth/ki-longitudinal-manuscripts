\documentclass[9pt,]{article}
\usepackage[]{mathpazo}
\usepackage{amssymb,amsmath}
\usepackage{ifxetex,ifluatex}
\usepackage{fixltx2e} % provides \textsubscript
\ifnum 0\ifxetex 1\fi\ifluatex 1\fi=0 % if pdftex
  \usepackage[T1]{fontenc}
  \usepackage[utf8]{inputenc}
\else % if luatex or xelatex
  \ifxetex
    \usepackage{mathspec}
  \else
    \usepackage{fontspec}
  \fi
  \defaultfontfeatures{Ligatures=TeX,Scale=MatchLowercase}
\fi
% use upquote if available, for straight quotes in verbatim environments
\IfFileExists{upquote.sty}{\usepackage{upquote}}{}
% use microtype if available
\IfFileExists{microtype.sty}{%
\usepackage{microtype}
\UseMicrotypeSet[protrusion]{basicmath} % disable protrusion for tt fonts
}{}
\usepackage[margin=1in]{geometry}
\usepackage{hyperref}
\hypersetup{unicode=true,
            pdfborder={0 0 0},
            breaklinks=true}
\urlstyle{same}  % don't use monospace font for urls
\usepackage{graphicx,grffile}
\makeatletter
\def\maxwidth{\ifdim\Gin@nat@width>\linewidth\linewidth\else\Gin@nat@width\fi}
\def\maxheight{\ifdim\Gin@nat@height>\textheight\textheight\else\Gin@nat@height\fi}
\makeatother
% Scale images if necessary, so that they will not overflow the page
% margins by default, and it is still possible to overwrite the defaults
% using explicit options in \includegraphics[width, height, ...]{}
\setkeys{Gin}{width=\maxwidth,height=\maxheight,keepaspectratio}
\IfFileExists{parskip.sty}{%
\usepackage{parskip}
}{% else
\setlength{\parindent}{0pt}
\setlength{\parskip}{6pt plus 2pt minus 1pt}
}
\setlength{\emergencystretch}{3em}  % prevent overfull lines
\providecommand{\tightlist}{%
  \setlength{\itemsep}{0pt}\setlength{\parskip}{0pt}}
\setcounter{secnumdepth}{0}
% Redefines (sub)paragraphs to behave more like sections
\ifx\paragraph\undefined\else
\let\oldparagraph\paragraph
\renewcommand{\paragraph}[1]{\oldparagraph{#1}\mbox{}}
\fi
\ifx\subparagraph\undefined\else
\let\oldsubparagraph\subparagraph
\renewcommand{\subparagraph}[1]{\oldsubparagraph{#1}\mbox{}}
\fi

%%% Use protect on footnotes to avoid problems with footnotes in titles
\let\rmarkdownfootnote\footnote%
\def\footnote{\protect\rmarkdownfootnote}

%%% Change title format to be more compact
\usepackage{titling}

% Create subtitle command for use in maketitle
\providecommand{\subtitle}[1]{
  \posttitle{
    \begin{center}\large#1\end{center}
    }
}

\setlength{\droptitle}{-2em}

  \title{}
    \pretitle{\vspace{\droptitle}}
  \posttitle{}
    \author{}
    \preauthor{}\postauthor{}
    \date{}
    \predate{}\postdate{}
  

\begin{document}

\hypertarget{supplementary-information-for-longitudinal-analyses-of-early-childhood-stunting-in-low-resource-settings}{%
\section{\texorpdfstring{Supplementary information for
\emph{Longitudinal analyses of early childhood stunting in low-resource
settings}}{Supplementary information for Longitudinal analyses of early childhood stunting in low-resource settings}}\label{supplementary-information-for-longitudinal-analyses-of-early-childhood-stunting-in-low-resource-settings}}

\raggedright

\textbf{Recommended citation:} Benjamin-Chung J, et al.~2020.
Longitudinal analyses of early childhood stunting in low-resource
settings. \emph{Journal Name}. doi.

\hypertarget{sensitivity-analysis-using-fixed-effects}{%
\section{Sensitivity analysis using fixed
effects}\label{sensitivity-analysis-using-fixed-effects}}

The primary analyses presented in this manuscript pooled across
individual studies using random effects. Inferences about estimates from
fixed effects models are restricted to only the included
studies.\footnote{Hedges, L. V. \& Vevea, J. L. Fixed- and
  random-effects models in meta-analysis. Psychol. Methods 3, 486--504
  (1998).} The random effects approach was more conservative in the
presence of study heterogeneity, as evidenced by larger confidence
intervals around each point estimates. Overall, the inference from
results produced by each method was similar.

\hypertarget{age-specific-prevalence}{%
\section{1. Age-specific prevalence}\label{age-specific-prevalence}}

\includegraphics{/Users/jadederong/Documents/CRG/ki/ki-longitudinal-manuscripts/6-shiny-app/figures/stunting/fig-stunt-2-prev-overall_region--allage-primary.png}
\textbf{Figure 1a. Age-specific stunting prevalence overall and
stratified by region and pooled using random effects. } Age-specific
stunting prevalence overall and stratified by region. The ``Overall''
panel includes 22 studies (N=53,194); the ``Africa'' panel includes 7
studies (N=15,756); the ``Latin America'' panel includes 5 studies
(N=2,205); and the ``South Asia'' panel includes 15 studies (N=28,492).
Results were estimated from data in up to 23 cohorts. The number of
children per estimate ranged from approximately 6,000 to 1.7 million.
Vertical bars indicate 95\% confidence intervals.

\includegraphics{/Users/jadederong/Documents/CRG/ki/ki-longitudinal-manuscripts/6-shiny-app/figures/stunting/fig-stunt-2-prev-overall_region--allage-fe.png}
\textbf{Figure 1b. Age-specific stunting prevalence overall and
stratified by region and pooled using fixed effects. } Results include
data from cohorts that measured children at least quarterly. We pooled
estimates across studies using inverse variance weighted fixed effects
models. Results shown in both panels were estimated from data in up to
23 cohorts. The number of children per estimate ranged from
approximately 6,000 to 1.7 million. Vertical bars indicate 95\%
confidence intervals.

\hypertarget{age-specific-incidence}{%
\section{2. Age-specific incidence}\label{age-specific-incidence}}

\includegraphics{/Users/jadederong/Documents/CRG/ki/ki-longitudinal-manuscripts/6-shiny-app/figures/stunting/fig-stunt-2-inc-overall_region--allage-primary.png}
\textbf{Figure 2a. Age-specific stunting incidence (cumulative and new
cases) overall and stratified by region and pooled using random
effects.} ) ``0-3'' includes ages from age 2 days up to 3 months. The
``Overall'' panel includes 23 studies (N=60,356); the ``Africa'' panel
includes 7 studies (N=19,859); the ``Latin America'' panel includes 5
studies (N=2,163); and the ``South Asia'' panel includes 15 studies
(N=30,358). Results shown in both panels include data from cohorts that
measured children at least quarterly. We pooled estimates across studies
using random effects models fit with restricted maximum likelihood
estimation. Results were estimated from data in up to 23 cohorts. The
number of children per estimate ranged from approximately 6,000 to 1.7
million. Vertical bars indicate 95\% confidence intervals.

\includegraphics{/Users/jadederong/Documents/CRG/ki/ki-longitudinal-manuscripts/6-shiny-app/figures/stunting/fig-stunt-2-inc-overall_region--allage-fe.png}
\textbf{Figure 2b. Age-specific stunting incidence (cumulative and new
cases) overall and stratified by region and pooled using fixed effects.}
Results include data from cohorts that measured children at least
quarterly. We pooled estimates across studies using inverse variance
weighted fixed effects models. Results shown in both panels were
estimated from data in up to 23 cohorts. The number of children per
estimate ranged from approximately 6,000 to 1.7 million. Vertical bars
indicate 95\% confidence intervals.

\hypertarget{relationship-between-laz-and-stunting-status}{%
\section{3. Relationship between LAZ and stunting
status}\label{relationship-between-laz-and-stunting-status}}

FIGURE TO BE ADDED

\hypertarget{r-bar-re-echo-false-include_graphicspaste0fig_dir-stuntingfig-stunt-2-flow-overall--allage-primary.png}{%
\section{\texorpdfstring{\texttt{\{r\ bar-re,\ echo\ =\ FALSE\}\ \#\ include\_graphics(paste0(fig\_dir,\ "stunting/fig-stunt-2-flow-overall-\/-allage-primary.png"))\ \#}}{\{r bar-re, echo = FALSE\} \# include\_graphics(paste0(fig\_dir, "stunting/fig-stunt-2-flow-overall-\/-allage-primary.png")) \#}}\label{r-bar-re-echo-false-include_graphicspaste0fig_dir-stuntingfig-stunt-2-flow-overall--allage-primary.png}}

\textbf{Relationship between LAZ and stunting status pooled using fixed
effects models.} Percentage of children stunted and not stunted by age.
``Never stunted'': children with LAZ \(\ge\) --2 at previous ages and
the current age. ``Still not stunted'': children who were previously
recovered with LAZ \(\ge\) --2 at the current age. ``No longer
stunted'': children with LAZ \textless{} --2 at the previous age and LAZ
\(\ge\)--2 at the current age. ``Newly stunted'': children whose LAZ was
previously always \(\ge\) --2 and with LAZ \textless{} --2 at the
current age. ``Stunting relapse'': children who were previously stunted
with LAZ \(\ge\) --2 at the previous age and LAZ \textless{} --2 at the
current age. ``Still stunted'': children whose LAZ was \textless{} --2
at the previous and current age. Analyses in b and c includes data from
12 cohorts with at least monthly measurement, 10 countries, and 11,394
children. Results shown in all panels were pooled across studies using
inverse variance weighted fixed effects models and include data subset
to ages up to 15 months because in most cohorts, measurements were less
frequent above 15 months.

FIGURE TO BE ADDED

\textbf{Relationship between LAZ and stunting status pooled using fixed
effects models.} Percentage of children stunted and not stunted by age.
``Never stunted'': children with LAZ \(\ge\) --2 at previous ages and
the current age. ``Still not stunted'': children who were previously
recovered with LAZ \(\ge\) --2 at the current age. ``No longer
stunted'': children with LAZ \textless{} --2 at the previous age and LAZ
\(\ge\)--2 at the current age. ``Newly stunted'': children whose LAZ was
previously always \(\ge\) --2 and with LAZ \textless{} --2 at the
current age. ``Stunting relapse'': children who were previously stunted
with LAZ \(\ge\) --2 at the previous age and LAZ \textless{} --2 at the
current age. ``Still stunted'': children whose LAZ was \textless{} --2
at the previous and current age. Analyses in b and c includes data from
12 cohorts with at least monthly measurement, 10 countries, and 11,394
children. Results shown in all panels were pooled across studies using
inverse variance weighted fixed effects models and include data subset
to ages up to 15 months because in most cohorts, measurements were less
frequent above 15 months.

\hypertarget{linear-growth-velocity}{%
\section{4. Linear growth velocity}\label{linear-growth-velocity}}

FIGURE TO BE ADDED

\hypertarget{r-bar-re-echo-false-include_graphicspaste0fig_dir-stuntingfig-stunt-2-vel-overall--allage-primary.png}{%
\section{\texorpdfstring{\texttt{\{r\ bar-re,\ echo\ =\ FALSE\}\ \#\ \#\ include\_graphics(paste0(fig\_dir,\ "stunting/fig-stunt-2-vel-overall-\/-allage-primary.png"))\ \#}}{\{r bar-re, echo = FALSE\} \# \# include\_graphics(paste0(fig\_dir, "stunting/fig-stunt-2-vel-overall-\/-allage-primary.png")) \#}}\label{r-bar-re-echo-false-include_graphicspaste0fig_dir-stuntingfig-stunt-2-vel-overall--allage-primary.png}}

\textbf{Figure 4a. Within-child difference in length in centimeters per
month stratified by age and sex pooled using random effects.} Dotted
black line indicates 15th percentile of the WHO Growth Velocity
Standards; dashed black line indicates the 25th percentile; solid black
line indicates the 50th percentile. Colored lines indicate and vertical
bars indicate 95\% confidence intervals for ki cohorts. Using pooled
random effects models, differences were statistically significant
between sexes at ages 0-3, 3-6, 12-15, 15-18, and 18-21 months.

FIGURE TO BE ADDED

\textbf{Figure 4b. Within-child difference in length in centimeters per
month stratified by age and sex pooled using fixed effects.} Dotted
black line indicates 15th percentile of the WHO Growth Velocity
Standards; dashed black line indicates the 25th percentile; solid black
line indicates the 50th percentile. Colored lines indicate and vertical
bars indicate 95\% confidence intervals for ki cohorts.

FIGURE TO BE ADDED

\textbf{Figure 4c. Within-child difference in length-for-age Z-score per
month by age and sex pooled using random effects.} Using pooled random
effects models, differences were statistically significant between sexes
at ages 0-3, 3-6, 6-9, 9-12, 18-21, and 21-24 months.

FIGURE TO BE ADDED

\textbf{Figure 4d. Within-child difference in length-for-age Z-score per
month by age and sex. }

FIGURE TO BE ADDED

\textbf{Figure 4e. Mean length-for-age Z-score by age and sex pooled
using random effects.} Differences by sex were statistically significant
in each age group. Results shown in all panels were derived from 23 ki
cohorts in 15 countries that measured children at least quarterly (n =
77,129 children) and pooled using random effects models fit with
restricted maximum likelihood estimation.

FIGURE TO BE ADDED

\textbf{Figure 4f. Mean length-for-age Z-score by age and sex.} Results
shown in all panels were derived from 23 ki cohorts in 15 countries that
measured children at least quarterly (n = 77,129 children) and pooled
using inverse variance weighted fixed effects models. 95\% confidence
intervals are included in plots but for certain estimates are not
visible because they are very close to the pooled point estimate.


\end{document}
